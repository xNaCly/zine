\documentclass[twocolumn, parskip=half, 10pt, a4paper]{article}

\usepackage{helvet}
\usepackage[T1]{fontenc}
\usepackage{inconsolata}
\usepackage{amsmath}
\renewcommand{\familydefault}{\sfdefault}

\usepackage[margin=2.2cm]{geometry}
\usepackage[hidelinks=false]{hyperref}
\usepackage{minted}
\usemintedstyle{xcode}
\usepackage{float}
\usepackage{graphicx}
\usepackage{hyperref}

\setlength{\parindent}{0cm}


\title{On Byte code Interpreters and their pipeline}

\begin{document}
    \pagenumbering{gobble}
    \subsection*{On Byte code Interpreters and their pipeline}

    To execute a list of instructions, a program, called an interpreter, is
    used. Interpreters are distinguished into categories depending on their
    execution strategy. Common examples are tree walking interpreters, the
    other strategy most modern interpreters use are byte code interpreters. To
    take a look at byte code interpretation with a register based approach, we
    need to establish a baseline of terms we use a lot in this article: byte
    code refers to a byte representing an instruction for the byte code virtual
    machine, such as \texttt{0x1} (could represent any instruction). A byte
    code virtual machine takes this byte, interprets it (could be an argument
    or could be an instruction) and executes the instruction.

    \begin{figure}[H]
        \centering
        \mintinline{javascript}{125.9*15+3.1415}
    \end{figure}

    Consider the above arithmetic expression. How would you represent this
    inherently tree like execution priority into a linear list of bytes? - We
    will do a deep dive into the interpretation pipeline in the next section,
    thereby answering this question. Most snippets will be in rust, refer to
    the
    rust\footnote{\href{https://github.com/xNaCly/calcrs}{https://github.com/xNaCly/calcrs}}
    or the go
    implementation\footnote{\href{https://github.com/xNaCly/calculator}{https://github.com/xNaCly/calculator}}
    if you are interested in the complete projects.

    \paragraph*{Interpretation pipeline} To evaluate and therefore interpret
    the example, our interpreter has to convert the expression into a machine
    readable format. The next step is parse this stream of tokens into a
    representation including expression precedence, known as an abstract syntax
    tree. Afterwards we compile this ast into instructions encoded as bytes.

    \subparagraph*{Tokenisation} Refers to the process of converting a stream
    of characters, such as \texttt{125.9*15} into tokens with meaning, taking a
    look at our example we can at first notate the token kinds we will need and
    afterwards define a list of tokens representing our example.

        \begin{minted}{rust}
enum Token {Plus, Asteriks, Number(f64)}
vec![Token::Number(125.9),Token::Asteriks,
Token::Number(15),Token::Plus, 
Token::Number(3.14)];
        \end{minted}

    \subparagraph*{Parsing}
    \subparagraph*{Compilation}
    \subparagraph*{Evaluation}
\end{document}

