\documentclass[twocolumn, parskip=half, 10pt, a4paper]{article}

\usepackage{helvet}
\usepackage[T1]{fontenc}
\usepackage{amsmath}
\renewcommand{\familydefault}{\sfdefault}

\usepackage[margin=2.5cm, foot=1cm]{geometry}
\usepackage[hidelinks=false]{hyperref}
\usepackage{minted}
\usemintedstyle{xcode}
\usepackage{float}
\usepackage{graphicx}

\setlength{\parindent}{0cm}


\title{On HashMaps and their Implementation}

\begin{document}
    \pagenumbering{gobble}
    \subsection*{On HashMaps and Implementations}

    \subsubsection*{Use case}

    \subsubsection*{Hashes and Hashing functions}

    A hash function $f$ maps a given input: $f(i) \rightarrow h$, $i$ to a hash
    $h$. $f(i)$ has to always compute to $h$ for the same $i$, otherwise the
    map would store values with the same key at the differing location. To keep
    map access $O(1)$ and map insert $O(n)$, the hash function computes to an
    integer. This integer is then used to index into an the underlying array.

    \subsubsection*{Performance and the load factor} 

    \subsubsection*{Dealing with Collisions}

    \subsubsection*{Naiive Implementation}
\end{document}

