\documentclass[twocolumn, parskip=half, 10pt, a4paper]{article}

\usepackage{helvet}
\usepackage[T1]{fontenc}
\usepackage{inconsolata}
\usepackage{amsmath}
\renewcommand{\familydefault}{\sfdefault}

\usepackage[margin=2.2cm]{geometry}
\usepackage[hidelinks=false]{hyperref}
\usepackage{minted}
\usemintedstyle{xcode}
\usepackage{float}
\usepackage{graphicx}
\usepackage{hyperref}

\setlength{\parindent}{0cm}


\title{On HashMaps and their Implementation}

\begin{document}
    \pagenumbering{gobble}
    \subsection*{On HashMaps and Implementations}

    \subsubsection*{Use case}

    \subsubsection*{Hashes and Hashing functions}

    A hash function $f$ maps a given input: $f(i) \rightarrow h$, $i$ to a hash
    $h$. $f(i)$ has to always compute to $h$ for the same $i$, otherwise the
    map would store values with the same key at different locations. To keep
    map access $O(1)$ and map insert $O(n)$, the hash function computes to an
    integer. This integer is then used to index into an the underlying array,
    instead of iterating a list or an array until the desired key is found. 

    Lets take a look at some common hashing applications: Java hashes strings
    by summing the characters of the string, while each is xored with the
    length minus the index of the character
    \footnote{\href{https://docs.oracle.com/javase/8/docs/api/java/lang/String.html\#hashCode}{String.hashCode()}}.
    \begin{minted}{java}
var s = "Hello World";
var h = 0;
for (int i = 0; i < s.length(); i++) {
    h += s.codePointAt(i) * 31 
        ^ (s.length() - (i+1));
}
    \end{minted}

    We will use a similar but better algorithm for hashing our key strings:
    fnv1-a.

    \subsubsection*{Performance and the load factor} 

    \subsubsection*{Dealing with Collisions}

    \subsubsection*{Naiive Implementation}
\end{document}

