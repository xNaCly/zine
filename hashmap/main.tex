\documentclass[twocolumn, parskip=half, 10pt, a4paper]{article}

\usepackage{helvet}
\usepackage[T1]{fontenc}
\usepackage{inconsolata}
\usepackage{amsmath}
\renewcommand{\familydefault}{\sfdefault}

\usepackage[margin=2.2cm]{geometry}
\usepackage[hidelinks=false]{hyperref}
\usepackage{minted}
\usemintedstyle{xcode}
\usepackage{float}
\usepackage{graphicx}
\usepackage{hyperref}

\setlength{\parindent}{0cm}


\title{On Hash maps and their shortest implementation possible}

\begin{document}
    \pagenumbering{gobble}
    \subsection*{On Hash maps and their shortest implementation possible}

    \paragraph*{About Hash maps}

    Explaining hash maps, their implementation and showing a very short but
    functioning implementation in c.

    Hash maps are the backbone of fast running programs. Hash maps power
    caches, make searching really fast (for certain workloads faster than
    search trees), they allow databases to create indexes for really fast
    lookups and they are used to create sets.

    \paragraph*{Hashes and Hashing functions}

    A hash function always computes the same integer for the same input, called
    a hash. This integer is then used to index into the underlying array of the
    map. If two differing inputs compute to the same hash, a hash collision
    occurs - with this collision can be dealt by storing a list of elements at
    the location the hash points to, thus allowing for more than one element
    for each hash
    \footnote{\href{https://en.wikipedia.org/wiki/Hash_collision}{https://en.wikipedia.org/wiki/Hash\_collision}}.

    Lets take a look at some common hashing applications: Java hashes strings
    by summing the characters of the string, while each is xored with the
    length. 
    \footnote{\href{https://docs.oracle.com/javase/8/docs/api/java/lang/String.html\#hashCode}{https://docs.oracle.com/javase/8/docs/api/java/lang/String.html}}.
    \begin{minted}{java}
var s = "Hello World";
for (int i = 0, h = 0; i < s.length(); i++)
    h += s.codePointAt(i)*31 
        ^ (s.length()-i);
    \end{minted}

    We will use a similar, but different algorithm for hashing our key strings:
    fnv-1a
    \footnote{\href{https://en.wikipedia.org/wiki/Fowler-Noll-Vo_hash_function}{https://en.wikipedia.org/wiki/Fowler-Noll-Vo\_hash\_function}}.
    The key of fnva-1a is to start with a default value for the hash, called
    the base, modify it by xoring it with the current character and then
    multiplying it with a prime number. On that basis we can create the first
    function of our naive implementation, \texttt{hash()} to hash our string
    keys:

    \begin{minted}{c}
const size_t BASE = 0x811c9dc5;
const size_t PRIME = 0x01000193;
size_t hash(Map *m, char *str) {
    size_t initial = BASE;
    while(*str) initial ^= *str++ * PRIME;
    return initial & (m->cap - 1);
}
    \end{minted}

    The first things to notice, is the two constants required by
    fnva-1a, the parameter of the hash function of the \texttt{Map}
    type and the bitwise and in the return statement. The \texttt{m}
    parameter is used specifically in combination with the bitwise
    \texttt{\&} to restrict the resulting hash to the size of the
    underlying array, thus eliminating out of bounds errors - this way
    of computing modulo is faster than \texttt{initial \% (m->cap-1)},
    but only works for the cap being a power of two. We
    control the size of the map, thus we can keep this in mind.
    \subsubsection*{Map Initialisation}

    The Map structure contains the capacity, the size and the array of buckets,
    each bucket containing \mintinline{c}{void *}. This type can also just be a
    value, such as a double. C however allows for erasing the type of a pointer
    by casting it: \mintinline{c}{(void *)p}. Therefore this map can contain
    any pointer and does not assume ownership over the value itself - the
    downside is, the user has to cast the inserted and extracted pointers,
    while keeping track of their lifetimes.

    \begin{minted}{c}
typedef struct Map { 
    size_t size;
    size_t cap;
    void **buckets; 
} Map;
    \end{minted}

    The \texttt{Map} is initialised with a size of $0$, the defined cap and by
    allocating the buckets. We check for allocation failures with the
    assertion.

    \begin{minted}{c}
Map init(size_t cap) {
  Map m = {0, cap};
  m.buckets = malloc(sizeof(void *) * m.cap);
  assert(m.buckets != NULL);
  return m;
}
    \end{minted}
    \paragraph*{Pointer Insertion}

    Inserting a pointer into the map consists of incrementing the size field,
    computing the hash and assigning the element at the index to the pointer we
    want to insert:

    \begin{minted}{c}
void put(Map *m, char *str, void *value) {
  m->size++;
  m->buckets[hash(m, str)] = value;
}
    \end{minted}

    \paragraph*{Pointer Extraction}
    Extracting a pointer works the same way as the insertion: computing the
    hash and returning the value at the index:


    \begin{minted}{c}
void *get(Map *m, char *str) {
    return m->buckets[hash(m, str)]; 
}
    \end{minted}

    \paragraph*{Usage Example}

    The callee of the map functions can even insert pointers to stack
    variables, even if they do not outlive the scope. They also have
    to free the allocated bucket array.

    \begin{minted}{c}
int main(void) {
  Map m = init(1024);
  double d1 = 25.0;
  put(&m, "key", (void *)&d1);
  printf("key=%f\n", *(double *)get(&m, "key"));
  free(m.buckets);
  return EXIT_SUCCESS;
}
    \end{minted}
\end{document}

