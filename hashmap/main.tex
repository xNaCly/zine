\documentclass[twocolumn, parskip=half, 10pt, a4paper]{article}

\usepackage{helvet}
\usepackage[T1]{fontenc}
\usepackage{amsmath}
\renewcommand{\familydefault}{\sfdefault}

\usepackage[margin=2.5cm, foot=1cm]{geometry}
\usepackage[hidelinks=false]{hyperref}
\usepackage{minted}
\usemintedstyle{xcode}
\usepackage{float}
\usepackage{graphicx}

\setlength{\parindent}{0cm}


\title{On HashMaps and their Implementation}

\begin{document}
    \pagenumbering{gobble}
    \subsection*{On HashMaps and Implementations}

    \subsubsection*{Use case}

    Hash maps are the backbone of fast running programs. They are used to
    create caches, make searching really fast (for certain workloads faster
    than search trees), index databases for fast lookups and to create sets.

    \subsubsection*{Hashes and Hashing functions}

    A hash function $f$ maps a given input: $f(i) \rightarrow h$, $i$ to a hash
    $h$. $f(i)$ has to always compute to $h$ for the same $i$, otherwise the
    map would store values with the same key at different locations. To keep
    map access $O(1)$ and map insert $O(n)$, the hash function computes to an
    integer. This integer is then used to index into an the underlying array,
    instead of iterating a list or an array until the desired key is found. 

    Lets take a look at some common hashing applications: Java hashes strings
    by summing the characters of the string, while each is xored with the
    length minus the index of the character
    \footnote{\href{https://docs.oracle.com/javase/8/docs/api/java/lang/String.html\#hashCode}{String.hashCode()}}.
    \begin{minted}{java}
var s = "Hello World";
var h = 0;
for (int i = 0; i < s.length(); i++) {
    h += s.codePointAt(i) * 31 
        ^ (s.length() - (i+1));
}
    \end{minted}

    We will use a similar, but different algorithm for hashing our key strings:
    fnv-1a
    \footnote{\href{https://en.wikipedia.org/wiki/Fowler-Noll-Vo_hash_function}{Fnv
    Hashing Wikipedia}}. The key of fnva-1a is to start with a large basis for
    the hash, called the base, and modify it by xoring it with the current
    character and multiplying it with a prime number, thus we can create the
    first function of our naive implementation, \texttt{hash()}:

    \begin{minted}{c}
const size_t BASE = 0x811c9dc5;
const size_t PRIME = 0x01000193;
size_t hash(Map *m, char *str) {
    size_t initial = BASE;
    while(*str) {
        initial ^= *str++;
        initial *= PRIME;
    }
    return initial & (m->cap - 1);
}
    \end{minted}

    The first things to notice, is the two constants required by fnva-1a, the
    parameter of the hash function of the \texttt{Map} type and the bitwise and
    in the return statement. The \texttt{m} parameter is used specifically in
    combination with the bitwise \texttt{\&} to restrict the resulting hash to
    the size of the underlying array, thus eliminating out of bounds errors.

    \subsubsection*{Map Initialisation}

    \begin{minted}{c}
typedef struct Map { 
    size_t size;
    size_t cap;
    void **buckets; 
} Map;
    \end{minted}

    \subsubsection*{Pointer Insertion}
    \subsubsection*{Pointer Extraction}
    \subsubsection*{Excursion: Hash Collisions}
    \subsubsection*{Excursion: Performance}
\end{document}

